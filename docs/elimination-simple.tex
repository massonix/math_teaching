%-----------------------------------------------------------------------
% Beginning of proc-l-template.tex
%-----------------------------------------------------------------------
%
%     This is a topmatter template file for PROC for use with AMS-LaTeX.
%
%     Templates for various common text, math and figure elements are
%     given following the \end{document} line.
%
%%%%%%%%%%%%%%%%%%%%%%%%%%%%%%%%%%%%%%%%%%%%%%%%%%%%%%%%%%%%%%%%%%%%%%%%

%     Remove any commented or uncommented macros you do not use.

\documentclass{proc-l}

%     If you need symbols beyond the basic set, uncomment this command.
%\usepackage{amssymb}

%     If your article includes graphics, uncomment this command.
%\usepackage{graphicx}

%     If the article includes commutative diagrams, ...
%\usepackage[cmtip,all]{xy}


%     Update the information and uncomment if AMS is not the copyright
%     holder.

\copyrightinfo{2020}{}

\newtheorem{theorem}{Theorem}[subsection]
\newtheorem{lemma}[theorem]{Lemma}

\theoremstyle{definition}
\newtheorem{definition}[theorem]{\bf Definition}
\newtheorem{example}[theorem]{Example}
\newtheorem{xca}[theorem]{Exercise}

\theoremstyle{remark}
\newtheorem{remark}[theorem]{Remark}

\numberwithin{equation}{section}

%aliases

\newcommand{\comment}[1]{}
\newcommand{\R}{\mathbb{R}}
\newcommand{\rank}[1]{\textrm{rank}({#1})}
\newcommand{\im}[1]{\textrm{Im}({#1})}
\newcommand{\x}{\times}
\newcommand{\id}[1]{\textrm{Id}_{{#1}\times {#1}}}
\renewcommand{\ker}[1]{\textrm{ker}({#1})}
\renewcommand{\span}[1]{\textrm{span}\{ {#1} \}}

% set section default counter
\setcounter{section}{1}

% do not show page numbers
\pagenumbering{gobble}


\begin{document}

% \title[short text for running head]{full title}
\title{Understading linear maps\\ with Gauss-Jordan elimination}

%    Only \author and \address are required; other information is
%    optional.  Remove any unused author tags.

%    author one information
% \author[short version for running head]{name for top of paper}
% \author{Ramon Massoni}
\address{}
\curraddr{}
\email{}
\thanks{}

%    author two information
% \author{Ferran Mui\~nos}
\address{}
\curraddr{}
\email{}
\thanks{}

%    \subjclass is required.
%    \subjclass[2010]{Primary }

%     \date{\today}

%     \dedicatory{}

%    "Communicated by" -- provide editor's name; required.
%     \commby{}

%    Abstract is required.
\begin{abstract}
We provide a condensed proof of the fundamental theorem of linear algebra in a Gauss-Jordan elimination fashion.
\end{abstract}

\maketitle

\subsection{}
We denote $\R^{m \x n}$ the collection of all matrices of shape $m\x n$.

\subsection{}
Any matrix $A\in\R^{m\times n}$ is associated with a linear map that we can denote $f_A:\R^n\to\R^m$. 

\subsection{}
Column-wise elementary operations can be of either of three kinds:
\begin{enumerate}
\item Permutation of columns.
\item Replacing a column $c_i$ with $\lambda c_i$ with $\lambda\neq 0$.
\item Replacing a column $c_i$ with $c_i + \mu c_j$ for any $\mu$.
\end{enumerate}
We denote $e(A)$ the transform of a matrix $A$ by a given elementary operation $e$.

\subsection{}
Hereinafter our discussion will be based entirely on column-wise operations, so we will drop ``column-wise'' and ``column'' from all our statements whenever possible.

\subsection{}
Each elementary operation of $A$ is equivalent to multiplying $A$ by an appropriate matrix $E \in \R^{n\times n}$. Hence, we say that $E$ is the matrix representing the elementary operation $e$ whenever $e(A) = AE$ for all possible matrices $A$.

\subsection{}
Note that elementary operations preserve the rank, i.e., $\rank{A} = \rank{e(A)} = \rank{AE}$. Can you see why?

\subsection{}
In Gauss-Jordan elimination we always keep track of two matrices, that we can simply represent as a tuple $(A, B)$ where $A\in\R^{m\x n}$ and $B\in\R^{n\x n}$. We convene $(A, B) \sim (A', B')$ to mean that $(A', B')$ results from $(A, B)$ by applying the same sequence of elementary operations to both matrices.

\subsection{}
Gauss-Jordan elimination is the method whereby the pair $(A, \id{n})$ is transformed, by means of applying a sequence of elementary operations, into another pair $(L, B)$ where $L$ has a canonical form known as ``reduced echelon form''. This means that $L$ satisfies the following requirements:
\begin{enumerate}
\item $L$ is a lower column echelon matrix.
\item The leading entries of $L$ are $1$.
\item The leading entries of $L$ are the only non-zero entries in their row. 
\end{enumerate}
Following our notation: $(A, \id{n})\sim (L, B)$. 

\subsection{}
Given an input matrix $A$, Gauss-Jordan elimination leads to a unique matrix $L$ in reduced echelon form. We can stress this fact by defining a Gauss-Jordan algorithm $\textrm{GJ}: A\mapsto L(A)$ that accepts $A$ as input and returns its unique reduced column echelon form.

\subsection{Proposition}
If $(A, \id{n}) \sim (L, B)$ then $L=AB$.
\begin{proof}
Let $E = E_1\cdots E_k$ the product of the matrices encoding the elementary operations that have been applied. By definition $L = AE$ and $B = \id{n} E = E$, so it is clear that $B=E$, i.e., $B$ keeps track of all the elementary operations that have been applied to A that give $L$. It follows that $L=AB$.
\end{proof}

\subsection{}
Observe that $\rank{B} = n$. Can you figure out why?

\subsection{Proposition}
If $A$ is a square, full-rank matrix, then $L(A) = \id{n}$ and $B=A^{-1}$.
\begin{proof}
If $(A, \id{n}) \sim (\id{n}, B)$ for some matrix $B$, then we know by the previous discussion that $\id{n} = AB$, so $B=A^{-1}$.
\end{proof}

\subsection{}
For the next discussion, let $(A, \id{n}) \sim (L, B)$, with $L$ in reduced echelon form. Let $B=[\;B_1 \;| \ldots |\; B_n\;]$ and $L=[\;L_1\;|\ldots|\; L_r\;|\;0\;|\ldots|\;0\;]$ specified by their respective columns, where $L_1,\ldots,L_r$ are non-zero column vectors.

\subsection{Theorem}
$\rank{A} = r$, where $r$ is the number of non-zero columns in $L$. Moreover, $L_1,\ldots,L_r$ form a basis of the column space of $A$, denoted $C(A)$. Therefore, $\dim{C(A)} = r$.

\subsection{}
Observe that the last $n - r$ columns of $L$ are zero. What does this mean?

\subsection{Theorem}
$B_1, \ldots, B_{n-r}$ form a basis of the null-space of $A$, denoted $N(A)$.
\begin{proof}
The following are known facts:
\begin{enumerate}
\item $B_1, \ldots, B_n$ is a basis of $\R^n$; in particular, any subset is linearly independent.
\item The vectors $L_i = AB_i$ for each $1\leq i\leq r$ form a linearly independent set.
\item The vectors $L_i=AB_i$ are zero for $r + 1\leq i\leq n$. Consequently, 
\[
S=\span{B_i\;|\; r + 1\leq i \leq n} \} \subset N(A).
\]
\end{enumerate}
Let's check that $N(A)\subset S$. By (1) we can write any $v\in N(A)$ as a linear combination $v=\sum_{i=1}^n\lambda_i B_i$. From the fact that $v\in N(A)$ and (3), it follows that $0 = Av = \sum_{i=1}^r \lambda_i AB_i = \sum_{i=1}^r \lambda_i L_i$. By (2) this cannot hold unless $\lambda_i = 0$ for $i=1,\ldots, r$. Then $v\in S$.
\end{proof}

\subsection{}
It follows that $\dim{N(A)} = n - r$.


\subsection{Fundamental Theorem of Linear Algebra}
For any matrix $A\in\R^{m\times n}$ the following identity holds:
\[
\dim{N(A)} + \dim{C(A)} = n.
\]

%    Text of article.

%    Bibliographies can be prepared with BibTeX using amsplain,
%    amsalpha, or (for "historical" overviews) natbib style.
\bibliographystyle{amsplain}
%    Insert the bibliography data here.

\end{document}

\comment{

%%%%%%%%%%%%%%%%%%%%%%%%%%%%%%%%%%%%%%%%%%%%%%%%%%%%%%%%%%%%%%%%%%%%%%%%

%    Templates for common elements of a journal article; for additional
%    information, see the AMS-LaTeX instructions manual, instr-l.pdf,
%    included in the PROC author package, and the amsthm user's guide,
%    linked from http://www.ams.org/tex/amslatex.html .

%    Section headings
\section{}
\subsection{}

%    Ordinary theorem and proof
\begin{theorem}[Optional addition to theorem head]
% text of theorem
\end{theorem}

\begin{proof}[Optional replacement proof heading]
% text of proof
\end{proof}

%    Figure insertion; default placement is top; if the figure occupies
%    more than 75% of a page, the [p] option should be specified.
\begin{figure}
\includegraphics{filename}
\caption{text of caption}
\label{}
\end{figure}

%    Mathematical displays; for additional information, see the amsmath
%    user's guide, linked from http://www.ams.org/tex/amslatex.html .

% Numbered equation
\begin{equation}
\end{equation}

% Unnumbered equation
\begin{equation*}
\end{equation*}

% Aligned equations
\begin{align}
  &  \\
  &
\end{align}

%-----------------------------------------------------------------------
% End of proc-l-template.tex
%-----------------------------------------------------------------------
}