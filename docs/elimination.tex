%-----------------------------------------------------------------------
% Beginning of proc-l-template.tex
%-----------------------------------------------------------------------
%
%     This is a topmatter template file for PROC for use with AMS-LaTeX.
%
%     Templates for various common text, math and figure elements are
%     given following the \end{document} line.
%
%%%%%%%%%%%%%%%%%%%%%%%%%%%%%%%%%%%%%%%%%%%%%%%%%%%%%%%%%%%%%%%%%%%%%%%%

%     Remove any commented or uncommented macros you do not use.

\documentclass{proc-l}

%     If you need symbols beyond the basic set, uncomment this command.
%\usepackage{amssymb}

%     If your article includes graphics, uncomment this command.
%\usepackage{graphicx}

%     If the article includes commutative diagrams, ...
%\usepackage[cmtip,all]{xy}


%     Update the information and uncomment if AMS is not the copyright
%     holder.

\copyrightinfo{2020}{}

\newtheorem{theorem}{Theorem}[subsection]
\newtheorem{lemma}[theorem]{Lemma}

\theoremstyle{definition}
\newtheorem{definition}[theorem]{\bf Definition}
\newtheorem{example}[theorem]{Example}
\newtheorem{xca}[theorem]{Exercise}

\theoremstyle{remark}
\newtheorem{remark}[theorem]{Remark}

\numberwithin{equation}{section}

%aliases

\newcommand{\comment}[1]{}
\newcommand{\R}{\mathbb{R}}
\newcommand{\rank}[1]{\textrm{rank}({#1})}
\newcommand{\im}[1]{\textrm{Im}({#1})}
\newcommand{\x}{\times}
\renewcommand{\ker}[1]{\textrm{ker}({#1})}
\renewcommand{\span}[1]{\textrm{span}\{ {#1} \}}



\begin{document}

% \title[short text for running head]{full title}
\title{Understading linear maps with Gauss-Jordan elimination}

%    Only \author and \address are required; other information is
%    optional.  Remove any unused author tags.

%    author one information
% \author[short version for running head]{name for top of paper}
\author{Ramon Massoni}
\address{}
\curraddr{}
\email{}
\thanks{}

%    author two information
\author{Ferran Mui\~nos}
\address{}
\curraddr{}
\email{}
\thanks{}

%    \subjclass is required.
%    \subjclass[2010]{Primary }

%     \date{\today}

%     \dedicatory{}

%    "Communicated by" -- provide editor's name; required.
%     \commby{}

%    Abstract is required.
\begin{abstract}
We study some basic properties of linear maps by using column-wise Gauss-Jordan elimination.
\end{abstract}

\maketitle

\section{Gauss-Jordan elimination}
\subsection{}
Given an $m\x n$ matrix $A$, we denote $f_A:\R^n\to\R^m$ the linear map defined by $A$ in coordinates of the canonical basis. We denote $\R^{m \x n}$ the space of matrices of shape $m\x n$.

\subsection{}
We say that a column-wise Gauss-Jordan pair is a tuple of matrices $(A, B)$ where $A\in\R^{m\x n}$ and $B\in\R^{n\x n}$. Gauss-Jordan pairs are meant to represent the steps in the execution of Gauss-Jordan elimination.

\subsection{}
Column-wise elementary operations can be of either of three kinds:
\begin{enumerate}
\item Permutation of columns.
\item Replacing a column $c_i$ with $\lambda c_i$ where $\lambda$ is a non-zero scalar.
\item Replacing a column $c_i$ with $c_i + \mu c_j$ where $\mu$ is any scalar.
\end{enumerate}
We denote $e(A)$ the transform of a matrix $A$ by an elementary operation $e$.

\subsection{}
Hereinafter our discussion will be based entirely on column-wise operations, so we will drop ``column-wise'' from our statements.

\subsection{}
Elementary operations can be regarded as linear maps $\R^n \to \R^n$. We say that $E$ is the matrix representing the elementary operation $e$ whenever $e(A) = AE$ for all matrices $A\in\R^{m\x n}$.

\subsection{}
Elementary operations are injective as linear maps $\R^n \to \R^n$, hence $\rank{A} = \rank{e(A)} = \rank{AE}$, i.e., they preserve the rank.

\subsection{}
We can extend elementary operations to Gauss-Jordan pairs as follows
\[
e: (A, B) \mapsto (e(A), e(B)) = (AE, BE)
\]

\subsection{}
We denote $(A, B) \sim (A', B')$ the fact that $(A, B)$ can be transformed into $(A', B')$ by applying a sequence of elementary operations.

\subsection{}
Gauss-Jordan elimination is the method whereby the pair $(A, I_n)$ is transformed, by means of applying a sequence of elementary operations, into a pair $(L, B)$ with $L$ in reduced lower column echelon form, which means the following:
\begin{enumerate}
\item $L$ is a lower column echelon matrix.
\item The leading entries of $L$ are $1$.
\item The leading entries of $L$ are the only non-zero entries in their row. 
\end{enumerate}
Following our notation: $(A, I_n)\sim (L, B)$. 

\subsection{}
Given an input $A$, Gauss-Jordan elimination leads to a unique matrix $L$ in reduced lower echelon form. We can stress this fact by defining a Gauss-Jordan transform $L: A\mapsto L(A)$ that takes $A$ to the unique reduced lower column echelon form reachable by Gauss-Jordan elimination.

\subsection{}
If $(A, I_n) \sim (L, B)$ then $L=AB$.
\begin{proof}
Let $E = E_1\cdots E_k$ the product of the matrices encoding the elementary operations that have been applied. By definition $L = AE$ and $B = I_n E = E$, so it follows that $L=AB$.
\end{proof}

\subsection{}
Observe that $\rank{B} = \rank{I_n} = n$.

\subsection{}
Observe that if $A$ is a square, full-rank matrix, $L(A) = I_n$.

\subsection{}
If $A$ is a square, full-rank matrix, then $(A, I_n) \sim (I_n, A^{-1})$.
\begin{proof}
We know that $(A, I_n) \sim (I_n, B)$ for some matrix $B$. Then $I_n = AB$, so $B=A^{-1}$.
\end{proof}

\subsection{}
For the next discussion, let $(A, I_n) \sim (L, B)$, with $L$ in reduced lower echelon form. Let $B=[b_1 \ldots b_n]$ be the matrix $B$ specified by its columns $b_i$.

\subsection{}
The rank of $A$ is precisely the number of non-zero columns in $L$. Moreover, these columns are the generators of $\im{f_A}$. Since any subset of non-zero columns of $L$ is a linearly independent set, it follows that $\dim\im{f_A} = \rank{A}$.

\subsection{}
Observe that the last $n - \rank{A}$ columns of $L$ are zero.

\subsection{}
The last $n - \rank{A}$ columns of $B$ form a basis of the null-space $\ker{f_A}$. 
\begin{proof}
The following are known facts:
\begin{enumerate}
\item $b_1, \ldots, b_n$ is a basis of $\R^n$.
\item $\{Ab_i\;|\; 1\leq i\leq \rank{A}\}$ is a linearly independent set.
\item $Ab_i = 0$ for $\rank{A} + 1\leq i\leq n$. Consequently, 
\[
S=\span{b_i\;|\; \rank{A} + 1\leq i \leq n} \} \subset \ker{f_A}.
\]
\end{enumerate}
Let's check that $\ker{f_A} \subset S$. By (1) we can write any $v\in\ker{f_A}$ as $v=\sum\lambda_i b_i$. Using $v\in\ker{f_A}$ and (3), we have $0 = f_A (v) = \sum_{i\leq\rank{A}} \lambda_i Ab_i$. By (2) this cannot hold unless $\lambda_i = 0$ for $i=1,\ldots, \rank{A}$. Then $v\in S$.
\end{proof}

\subsection{}
It follows that $\dim{\ker{f_A}} = n - \rank{A}$.


\subsection{}
The following important result follows from all the discussion above. For any linear map $f_A:\R^n\to\R^m$ the following identity holds:
\[
\dim{\ker{f_A}} + \dim{\im{f_A}} = n.
\]

%    Text of article.

%    Bibliographies can be prepared with BibTeX using amsplain,
%    amsalpha, or (for "historical" overviews) natbib style.
\bibliographystyle{amsplain}
%    Insert the bibliography data here.

\end{document}

\comment{

%%%%%%%%%%%%%%%%%%%%%%%%%%%%%%%%%%%%%%%%%%%%%%%%%%%%%%%%%%%%%%%%%%%%%%%%

%    Templates for common elements of a journal article; for additional
%    information, see the AMS-LaTeX instructions manual, instr-l.pdf,
%    included in the PROC author package, and the amsthm user's guide,
%    linked from http://www.ams.org/tex/amslatex.html .

%    Section headings
\section{}
\subsection{}

%    Ordinary theorem and proof
\begin{theorem}[Optional addition to theorem head]
% text of theorem
\end{theorem}

\begin{proof}[Optional replacement proof heading]
% text of proof
\end{proof}

%    Figure insertion; default placement is top; if the figure occupies
%    more than 75% of a page, the [p] option should be specified.
\begin{figure}
\includegraphics{filename}
\caption{text of caption}
\label{}
\end{figure}

%    Mathematical displays; for additional information, see the amsmath
%    user's guide, linked from http://www.ams.org/tex/amslatex.html .

% Numbered equation
\begin{equation}
\end{equation}

% Unnumbered equation
\begin{equation*}
\end{equation*}

% Aligned equations
\begin{align}
  &  \\
  &
\end{align}

%-----------------------------------------------------------------------
% End of proc-l-template.tex
%-----------------------------------------------------------------------
}