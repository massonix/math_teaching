%-----------------------------------------------------------------------
% Beginning of proc-l-template.tex
%-----------------------------------------------------------------------
%
%     This is a topmatter template file for PROC for use with AMS-LaTeX.
%
%     Templates for various common text, math and figure elements are
%     given following the \end{document} line.
%
%%%%%%%%%%%%%%%%%%%%%%%%%%%%%%%%%%%%%%%%%%%%%%%%%%%%%%%%%%%%%%%%%%%%%%%%

%     Remove any commented or uncommented macros you do not use.

\documentclass{proc-l}

%     If you need symbols beyond the basic set, uncomment this command.
%\usepackage{amssymb}

%     If your article includes graphics, uncomment this command.
%\usepackage{graphicx}

%     If the article includes commutative diagrams, ...
%\usepackage[cmtip,all]{xy}


%     Update the information and uncomment if AMS is not the copyright
%     holder.

\copyrightinfo{2020}{}

\newtheorem{theorem}{Theorem}[subsection]
\newtheorem{lemma}[theorem]{Lemma}

\theoremstyle{definition}
\newtheorem{definition}[theorem]{\bf Definition}
\newtheorem{example}[theorem]{Example}
\newtheorem{xca}[theorem]{Exercise}

\theoremstyle{remark}
\newtheorem{remark}[theorem]{Remark}

\numberwithin{equation}{section}

%aliases

\newcommand{\comment}[1]{}
\newcommand{\R}{\mathbb{R}}
\newcommand{\rank}[1]{\textrm{rank}({#1})}
\newcommand{\im}[1]{\textrm{Im}({#1})}
\renewcommand{\ker}[1]{\textrm{ker}({#1})}



\begin{document}

% \title[short text for running head]{full title}
\title{Understading linear maps with Gauss-Jordan elimination}

%    Only \author and \address are required; other information is
%    optional.  Remove any unused author tags.

%    author one information
% \author[short version for running head]{name for top of paper}
\author{Ramon Massoni}
\address{}
\curraddr{}
\email{}
\thanks{}

%    author two information
\author{Ferran Mui\~nos}
\address{}
\curraddr{}
\email{}
\thanks{}

%    \subjclass is required.
%    \subjclass[2010]{Primary }

%     \date{\today}

%     \dedicatory{}

%    "Communicated by" -- provide editor's name; required.
%     \commby{}

%    Abstract is required.
\begin{abstract}
We study some basic properties of linear maps by using column-wise Gauss-Jordan elimination.
\end{abstract}

\maketitle

\section{Gauss-Jordan elimination}
\subsection{}
Given an $m\times n$ matrix $A$, we denote $f_A:\R^n\to\R^m$ the linear map defined by $A$ in coordinates of the canonical basis. We denote the space of matrices of shape $m\times n$ by $\R^m\times n$.

\subsection{}
For convenience we say that a pair of matrices $(A, B)$ where $A\in\R^{m\times n}$ and $B\in\R^{n\times n}$ is a (column-wise) Gauss-Jordan pair.

\subsection{}
Gauss-Jordan (column-wise) pairs intend to represent the steps in the Gauss-Jordan elimination procedure. The motivation is that each pair will represent the state of a Gauss-Jordan elimination $A$ is the matrix under study and $B$ keeps track of the column-wise elementary operations.

\subsection{}
Column-wise elementary operations can be of either of three kinds:
\begin{enumerate}
\item Permutation of columns.
\item Replacing a column $c_i$ with $\lambda c_i$ with $\lambda \neq 0$.
\item Replacing a column $c_i$ with $c_i + \mu c_j$ with $\mu$ any scalar.
\end{enumerate}

\subsection{}
Column-wise elementary operation can be encoded as linear maps $\R^n \to \R^n$ (endomorphisms of $\R^n$). Hence we can encode such operations via matrix multiplication. In particular, we can encode in this way any composition of elementary operations by simply multiplying their respective matrices.

\subsection{}
Such operations are equivalent to injective linear maps, hence they preserve the rank.

\subsection{}
We can apply column-wise elementary operations on pairs as follows $(A, B) \mapsto (g(A), g(B))$.

\subsection{}
Whenever $(A, B)$ is taken to $(A', B')$ by column-wise elementary operations, we denote it as $(A, B) \sim (A', B')$.

\subsection{}
Column-wise Gauss-Jordan elimination is the method whereby the (column-wise) Gauss-Jordan pair $(A, I_n)$ is transformed into another pair $(T, B)$ with $T$ in upper-column-echelon form by means of iteratively applying a sequence of column-wise elementary operations. Hence $(A, I_n)\sim (T, B)$. Without loss of generality

\subsection{}
If $(A, I_n) \sim (T, B)$ then $T=AB$.
\begin{proof}
Let $E$ be the matrix encoding a given column-wise elementary operation on columns. Then if $(T, B)$ has been obtained from $(A, I_n)$ by applying such operation, then $T=AE$ and $B=I_n E$ whence the claim follows. The general case follows by induction.
\end{proof}

\subsection{}
Observe that $\rank{B} = \rank{I_n} = n$.

\subsection{}
If $A$ is a square, full-rank matrix, then $(A, I_n) \sim (I_n, A^{-1})$.

\subsection{}
Let $(A, I_n) \sim (T, B)$, with $T$ in upper-column-echelon form. Let $B=[B_1 \ldots B_n]$. Then the last $n - \rank{A}$ columns of $T$ are zero, and $\ker{f_A}=\textrm{span}\left(B_j \;|\; j\in\{ \rank{A} + 1, \ldots, n\}\right)$. Since the columns of $B$ are linearly independent, then $\dim{\ker{f_A}} = n - \rank{A}$.

\subsection{}
The \rank{A} non-zero columns of $T$ form a basis of $\im{f_A}$, then $\dim{\im{f_A}} = \rank(A)$.

\subsection{}
The following important result follows from the discussion above. For any linear map $f_A:\R^n\to\R^m$ the following identity holds:
\[
\dim{\ker{f_A}} + \dim{\im{f_A}} = (n - \rank{A}) + \rank{A} = n.
\]

%    Text of article.

%    Bibliographies can be prepared with BibTeX using amsplain,
%    amsalpha, or (for "historical" overviews) natbib style.
\bibliographystyle{amsplain}
%    Insert the bibliography data here.

\end{document}

\comment{

%%%%%%%%%%%%%%%%%%%%%%%%%%%%%%%%%%%%%%%%%%%%%%%%%%%%%%%%%%%%%%%%%%%%%%%%

%    Templates for common elements of a journal article; for additional
%    information, see the AMS-LaTeX instructions manual, instr-l.pdf,
%    included in the PROC author package, and the amsthm user's guide,
%    linked from http://www.ams.org/tex/amslatex.html .

%    Section headings
\section{}
\subsection{}

%    Ordinary theorem and proof
\begin{theorem}[Optional addition to theorem head]
% text of theorem
\end{theorem}

\begin{proof}[Optional replacement proof heading]
% text of proof
\end{proof}

%    Figure insertion; default placement is top; if the figure occupies
%    more than 75% of a page, the [p] option should be specified.
\begin{figure}
\includegraphics{filename}
\caption{text of caption}
\label{}
\end{figure}

%    Mathematical displays; for additional information, see the amsmath
%    user's guide, linked from http://www.ams.org/tex/amslatex.html .

% Numbered equation
\begin{equation}
\end{equation}

% Unnumbered equation
\begin{equation*}
\end{equation*}

% Aligned equations
\begin{align}
  &  \\
  &
\end{align}

%-----------------------------------------------------------------------
% End of proc-l-template.tex
%-----------------------------------------------------------------------
}